{\color{gray}\hrule}
\begin{center}
\section{Kesimpulan}
\textbf{kesimpulan hasil praktikum}
\end{center}
{\color{gray}\hrule}
\vspace{0.5cm}
Pada pertemuan kedua, mahasisawa melakukan proses instalasi Apache Hadoop di lingkungan yang telah di persiapkan sebelumnya. Langkah-langkahnya termasuk membuat grup dan pengguna baru, menginstal Java, melakukan konfigurasi SSH, mengunduh dan mengekstrak Apache Hadoop, serta menyesuaikan hak akses. Setelah itu, kita melakukan konfigurasi berbagai file Hadoop dan membuat folder khusus untuk menyimpan data sementara dan direktori untuk Namenode/Datanode. Proses instalasi diakhiri dengan memformat HDFS. Hasil instalasi kemudian diverifikasi untuk memastikan bahwa semuanya berjalan dengan baik. Selain itu, kita juga melakukan beberapa konfigurasi pada file-file Hadoop agar memudahkan pemantauan ekosistem Hadoop yang telah diinstal. Secara keseluruhan, langkah-langkah ini membentuk dasar instalasi dan konfigurasi Apache Hadoop yang siap digunakan dalam pengolahan Big Data.