\begin{multicols}{2}
\tableofcontents
\section{Pendahuluan}
Hadoop adalah sebuah framework perangkat lunak open source yang digunakan untuk menyimpan dan memproses data secara terdistribusi. Framework ini dirancang untuk mengatasi masalah pemrosesan data yang sangat besar di lingkungan yang terdiri dari sejumlah besar komputer. Hadoop didasarkan pada model pemrograman MapReduce, yang dikembangkan oleh Google. Hadoop adalah sebuah framework perangkat lunak open source yang digunakan untuk menyimpan dan memproses data secara terdistribusi. Framework ini dirancang untuk mengatasi masalah pemrosesan data yang sangat besar di lingkungan yang terdiri dari sejumlah besar komputer. Hadoop didasarkan pada model pemrograman MapReduce, yang dikembangkan oleh Google.
\subsection{Latar Belakang}
Dilihat dari perkembangan teknologi informasi yang cepat, yang memberikan konsekuensi pertumbuhan data yang besar, dikenal sebagai Big Data. Big Data merupakan media penyimpanan data yang menawarkan ruang tak terbatas dan kemampuan untuk mengakomodasi serta memproses berbagai jenis data dengan sangat cepat. Dalam konteks ini, Hadoop Distributed File System (HDFS) menjadi sebuah media penyimpanan data utama dengan kapasitas data besar yang digunakan oleh Hadoop. Seiring dengan kebutuhan akan penyimpanan dan pemrosesan data yang semakin besar, Hadoop menjadi penting dalam menangani volume dan variasi data yang terus meningkat, terutama dari media sosial dan internet.
\subsection{Tujuan}
Praktikum ini bertujuan memperkenalkan langkah-langkah instalasi dan konfigurasi Apache Hadoop dalam lingkungan virtual menggunakan VirtualBox dan sistem operasi Ubuntu. Dengan tujuan utama memberikan pemahaman praktis tentang penggunaan Apache Hadoop dalam pengolahan Big Data, praktikum ini dimaksudkan untuk membekali mahasiswa dengan keterampilan instalasi dan konfigurasi yang mendalam.
\subsection{Tinjauan Pustaka}
Dalam praktikum ini, tinjauan pustaka diarahkan pada beberapa sumber kunci yang memberikan pemahaman mendalam kepada mahasiswa mengenai instalasi dan konfigurasi Apache Hadoop, serta konsep dasar yang melingkupinya. Buku "Hadoop: The Definitive Guide" oleh Tom White dari O'Reilly Media menjadi sumber utama yang memberikan wawasan terperinci tentang Apache Hadoop. Buku ini tidak hanya menjelaskan konsep fundamental Hadoop, tetapi juga memberikan panduan langkah demi langkah tentang instalasi dan konfigurasi. Selain itu, dokumentasi resmi dari situs web Apache Hadoop menjadi rujukan utama yang menyediakan informasi yang sahih dan terkini mengenai instalasi dan konfigurasi Hadoop. Tutorial dan dokumentasi resmi Ubuntu juga dijadikan acuan untuk mahasiswa dalam memahami langkah-langkah instalasi dan konfigurasi Ubuntu sebagai sistem operasi pada praktikum. Dengan menggabungkan sumber-sumber ini, diharapkan mahasiswa dapat memperoleh pemahaman yang holistik tentang Apache Hadoop, termasuk konfigurasi dan penerapannya dalam lingkungan virtual menggunakan VirtualBox dan Ubuntu.
\end{multicols}